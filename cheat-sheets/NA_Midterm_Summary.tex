\documentclass[10pt]{article} % 10pt 字体,紧凑
\usepackage[UTF8]{ctex}
\usepackage[utf8]{inputenc}
\usepackage{amsmath, amssymb, amsthm} % 数学符号和公式
\usepackage{geometry} % 页面设置
\usepackage{multicol} % 分栏
\usepackage{enumitem} % 紧凑列表
\usepackage{xcolor} % 颜色(可选,用于高亮)
\usepackage{mathtools} % 在导言区添加

% ===== 页面设置 =====
\geometry{a4paper, margin=0.5cm} % 小页边距,最大化利用空间
\setlength{\columnsep}{0.5cm} % 栏间距
\setlength{\parindent}{0pt} % 无缩进
\setlength{\parskip}{0pt} % 段落无间距

% % ===== 自定义命令 =====
% % 常用数学符号
% \newcommand{\R}{\mathbb{R}}
% \newcommand{\C}{\mathbb{C}}
% \newcommand{\N}{\mathbb{N}}
% \newcommand{\Z}{\mathbb{Z}}
% \newcommand{\Q}{\mathbb{Q}}

% % 常用运算符
% \newcommand{\diff}{\mathrm{d}}
% \newcommand{\T}{\mathsf{T}} % 转置

% % 矩阵命令
% \newcommand{\mat}[1]{\begin{bmatrix} #1 \end{bmatrix}}

% 定理环境(可选)
\newtheorem{theorem}{定理}

% ===== 文档开始 =====
\begin{document}
\footnotesize % 小字体,容纳更多内容
\begin{multicols}{2} % 分三栏,根据内容调整

% 你的内容从这里开始
\section*{1-绪论}

\subsection*{误差与有效数字}
\begin{itemize}[leftmargin=*]
\item \textbf{有效数字定义:}
若 $\lfloor \log_{10}(x^*) \rfloor = \textcolor{green}{m}$,$\lceil \log_{10}(2|x-x^*|) \rceil = \textcolor{green}{m} - \textcolor{blue}{n} + 1$,则称近似值 $x^*$ 有 \textbf{\textcolor{blue}{n} 位有效数字}
\end{itemize}

\section*{2-多项式插值}

\subsection*{① Lagrange 插值}

\textbf{Lagrange 插值法的误差分析}
设\textcolor{blue}{$f^{(n+1)}(x)$在$[a,b]$内存在},节点$a \leq x_0 < x_1 < \cdots < x_n \leq b$,$L_n(x)$是 Lagrange 插值多项式,则插值余项
\[
R_n(x) = \frac{f^{(n+1)}(\xi)}{(n+1)!}\omega_{n+1}(x)
\]

\begin{itemize}
\item 如果可以求出
$M_{n+1} = \max\limits_{a\leq x\leq b}|f^{(n+1)}(x)|$,
则插值多项式 $L(x)$ 逼近 $f(x)$ 的截断误差限是
\[
|R_n(x)| \leq \frac{M_{n+1}}{(n+1)!}|\omega_{n+1}(x)|
\]
\end{itemize}

\subsection*{②\ 差商与 Newton 插值}

\paragraph{逐次线性插值}
新的插值基函数
\begin{align*}
\omega_0 &= 1 \\
\omega_1 &= x - x_0 \\
\omega_2 &= (x - x_0)(x - x_1) \\
&\vdots \\
\omega_n &= (x - x_0)(x - x_1)\cdots(x - x_{n-1}) \\
& \Rightarrow \textcolor{red}{\omega_{n+1} = \omega_n \cdot (x - x_n)}
\end{align*}

\begin{minipage}{0.48\textwidth}
\begin{align*}
p_1(x) &= y_0\omega_0 + \textcolor{red}{\frac{y_1 - y_0}{x_1 - x_0}}\omega_1 \\
p_2(x) &= p_1(x) + \textcolor{red}{\frac{\frac{y_2 - y_0}{x_2 - x_0} - \frac{y_1 - y_0}{x_1 - x_0}}{x_2 - x_1}}\omega_2 \\
&\vdots \\
p_n(x) &= p_{n-1}(x) + \textcolor{red}{a_n}\omega_n
\end{align*}
\end{minipage}

\paragraph{差商}
\begin{itemize}
\item 已知函数 $f(x)$ 和节点 $x_0, x_1, \ldots, x_k$, 则
\begin{align*}
& f[x_0, x_k] = \frac{f(x_k)-f(x_0)}{x_k-x_0} \\
& f[x_0, x_1, x_k] = \frac{f[x_0, x_k] - f[x_0, x_1]}{x_k - x_1} \\
& f [x_0, x_1, \ldots, x_k] = \frac{\splitfrac{f[x_0, x_1, \ldots, x_{k-2}, x_k]}{- f[x_0, x_1, \ldots, x_{k-2}, x_{k-1}]}}{x_k - x_{k-1}}
\end{align*}

\item 用差商重新表示插值多项式
\begin{align*}
p_1(x) &= y_0 + \textcolor{red}{f[x_0, x_1]}\omega_1 \\
p_2(x) &= y_0 + \textcolor{red}{f[x_0, x_1]}\omega_1 + \textcolor{red}{f[x_0, x_1, x_2]}\omega_2
\end{align*}
\end{itemize}

\begin{align*}
f(x) = & f(x_0) + \textcolor{blue}{f[x_0, x]}(x - x_0) \\
 = & f(x_0) + \textcolor{blue}{\bigl(f[x_0, x_1] + f[x_0, x_1, x](x - x_1)\bigr)} \cdot (x - x_0) \\
 = & f(x_0) + f[x_0, x_1](x - x_0) + \textcolor{green}{f[x_0, x_1, x]} \cdot (x - x_0)(x - x_1) \\
 = & f(x_0) + f[x_0, x_1](x - x_0) + \bigl(\textcolor{green}{f[x_0, x_1, x_2]} \\
   & \textcolor{green}{- f[x_0, x_1, x_2, x](x - x_2)}\bigr) \cdot (x - x_0)(x - x_1) \\
 = & f(x_0) + f[x_0, x_1](x - x_0) \\
   & + f[x_0, x_1, x_2](x - x_0)(x - x_1) \\
   & + \textcolor{red}{f[x_0, x_1, x_2, x]}(x-x_0)(x-x_1)(x-x_2)
\end{align*}

\columnbreak

\fbox{\textbf{Newton 差商插值多项式}} \\[0.5ex]
\colorbox{yellow!20}{$
\begin{multlined}
N_n(x) = f[x_0] + f[x_0, x_1]\omega_1(x) + \cdots + f[x_0, \ldots, x_n]\omega_n(x)
\end{multlined}
$}

\fbox{\textbf{插值余项}} \\[0.5ex]
\colorbox{red!10}{$
\begin{multlined}
R_n(x) = f[x_0, x_1, \ldots, x_n, x] \cdot \omega_{n+1}(x)
\end{multlined}
$}

\paragraph{差商的性质}
\begin{enumerate}
\item 差商可以表示为函数值 $f(x_0), f(x_1), \ldots, f(x_k)$ 的线性组合
\[
f[x_0, x_1, \ldots, x_k] = \sum_{j=0}^{k} \frac{f(x_j)}{\prod_{i \neq j}(x_j - x_i)}
\]
证明(归纳法)

\item 差商与节点的排序无关,即差商具有对称性

\item 差商与阶导数之间的关系
\[
f[x_0, x_1, \ldots, x_k] = \frac{f^{(k)}(\xi)}{k!}
\]
思路:Lagrange 插值余项的定理

\item (作为泛函对$f$的)线性性

\item 对于不超过 $n$ 次的多项式 $f(x)$,其 $n$ 阶差商 $f[x_0, x_1, \ldots, x_n]$ 为常数($n$ 次项系数)
\end{enumerate}

\subsection*{等距节点插值公式 $\Rightarrow$ 差分}

\begin{itemize}
\item 为表示方便,引入不变算子 $I$ 和移位算子 $E$
\item 可以发现,向前差分算子 $\Delta = E - I$\\
同理可得,向后差分算子 $\nabla = I - E^{-1}$\\
\end{itemize}

\paragraph{差分的性质}
\begin{enumerate}
\item 各阶差分均可用函数值表示\\
$m$ 阶向前差分:$\Delta^m f_k = (E-I)^m f_k$

\item 可用各阶差分表示函数值\\
$f_{k+m} = (\Delta+I)^m f_k$

\item 差商与差分的关系\\
$x_k = x_0 + kh, \quad f[x_k, x_{k+1}, \ldots, x_{k+m}] = \dfrac{1}{m!}\dfrac{1}{h^m}\Delta^m f_k$

\item 向前差分 $\Rightarrow$ Newton 前插公式
\begin{align*}
N_n(x) &= N_n(x_0 + th) = f(x_0) + t\Delta f_0 + \binom{t}{2}\Delta^2 f_0 + \cdots + \binom{t}{n}\Delta^n f_0 \\[1ex]
R_n(x) &= \frac{f^{(n+1)}(\xi)}{(n+1)!} \binom{t}{n+1} h^{n+1}, \quad \xi \in (x_0, x_n)
\end{align*}
\end{enumerate}

\subsection*{③ Hermite 插值}

\textbf{Hermite 插值多项式}

\begin{itemize}
\item $2n + 1$ 次 Hermite 插值多项式
\begin{align*}
& H_{2n+1}(x) = \sum_{i=0}^{n} \left[y_i \alpha_i(x) + m_i \beta_i(x)\right] \\
& \alpha_i(x) = \left[1 - 2(x-x_i)\sum_{k \neq i}\frac{1}{x_i - x_k}\right]l_{i}^2(x) \\
& \beta_i(x) = (x - x_i)l_{i}^2(x)
\end{align*}

\item Hermite 插值余项
\[
R_{2n+1}(x) = \frac{f^{(2n+2)}(\xi)}{(2n+2)!}\omega_{n+1}^2(x)
\]

\item 两点三次 Hermite 插值
\begin{align*}
H_3(x) = & y_0\left(1+2\frac{x-x_0}{x_1-x_0}\right)\left(\frac{x-x_1}{x_0-x_1}\right)^2 \\
& + y_1\left(1+2\frac{x-x_1}{x_0-x_1}\right)\left(\frac{x-x_0}{x_1-x_0}\right)^2 \\
& + m_0(x-x_0)\left(\frac{x-x_1}{x_0-x_1}\right)^2 \\
& + m_1(x-x_1)\left(\frac{x-x_0}{x_1-x_0}\right)^2
\end{align*}

\item 插值余项
\[
R_3(x) = \frac{f^{(4)}(\xi)}{4!}(x-x_0)^2(x-x_1)^2
\]
\end{itemize}

\subsection*{④\ 分段插值}
\textbf{三次样条插值}

{如何求解三次样条插值函数?} \\

\textbf{思路 1:直接利用分段三次 Hermite 插值} \\
对 $k = 0, 1, \ldots, n - 1$,计算 $s_k(x)$ 的二阶导数,列出 $n+1$ 个方程,求解未知数 $m_0, m_1, \ldots, m_n$(一阶导数值)

\begin{align*}
s_k(x) = & y_k \left(1 + 2\frac{x - x_k}{x_{k+1} - x_k}\right)\left(\frac{x - x_{k+1}}{x_k - x_{k+1}}\right)^2 \\
& + y_{k+1}\left(1 + 2\frac{x-x_{k+1}}{x_k-x_{k+1}}\right)\left(\frac{x-x_k}{x_{k+1}-x_k}\right)^2 \\
& + \textcolor{red}{m_k}(x-x_k)\left(\frac{x - x_{k+1}}{x_k - x_{k+1}}\right)^2 \\
& + \textcolor{red}{m_{k+1}}(x-x_{k+1})\left(\frac{x-x_k}{x_{k+1}-x_k}\right)^2
\end{align*}

求二阶导数,得
\begin{align*}
s_k''(x) = & 2\frac{(2m_k+m_{k+1})(x-x_{k+1})+(m_k+2m_{k+1})(x-x_k)}{(x_{k+1}-x_k)^2} \\
& - 6\frac{y_{k+1}-y_k}{(x_{k+1}-x_k)^3}(x-x_{k+1}+x-x_k)
\end{align*}

\begin{align*}
\Rightarrow & s_k''(x_k) = 2\frac{3f[x_k,x_{k+1}]-2m_k-m_{k+1}}{x_{k+1}-x_k}, \\
& s_k''(x_{k+1}) = 2\frac{-3f[x_k,x_{k+1}]+2m_{k+1}+m_k}{x_{k+1}-x_k}
\end{align*}

代入二阶导数连续的条件 \\
$s_k''(x_{k+1})=s_{k+1}''(x_{k+1})$ \\
并利用 $h_k = x_{k+1} - x_k$ 化简,得
\begin{align*}
\frac{1}{h_k}\textcolor{red}{m_k} + 2\left(\frac{1}{h_k}+\frac{1}{h_{k+1}}\right)\textcolor{red}{m_{k+1}} + \frac{1}{h_{k+1}}\textcolor{red}{m_{k+2}} \\
= 3\left(\frac{f[x_k, x_{k+1}]}{h_k} + \frac{f[x_{k+1}, x_{k+2}]}{h_{k+1}}\right)
\end{align*}

$k = 0, 1, \ldots, n - 2$


\textbf{思路 2:基于 $S(x)$ 的二阶导数 $S''(x_k)=\textcolor{red}{M_k}$ ($k = 0, 1, \ldots, n$) 反推 $S(x)$}

由于 $s_k(x)$ 是三次多项式,可知 $s_k''(x)$ 是线性函数
\[
s_k''(x) = M_k \frac{x-x_{k+1}}{x_k-x_{k+1}} + M_{k+1}\frac{x-x_k}{x_{k+1}-x_k}
\]

对上式积分两次,可得
\begin{align*}
s_k(x) = & \frac{(x - x_{k+1})^3}{6(x_k - x_{k+1})}\textcolor{red}{M_k} + \frac{(x-x_k)^3}{6(x_{k+1}-x_k)}\textcolor{red}{M_{k+1}} \\
& + c_1 \frac{x-x_{k+1}}{x_k-x_{k+1}} + c_2 \frac{x-x_k}{x_{k+1}-x_k}
\end{align*}

代入插值条件 $s_k(x_k)=y_k$ 和 $s_k(x_{k+1})=y_{k+1}$,可确定积分常数 $c_1$ 和 $c_2$\\
整理后,得到
\begin{align*}
s_k(x) = & \frac{(x - x_{k+1})^3}{6(x_k - x_{k+1})}\textcolor{red}{M_k} + \frac{(x-x_k)^3}{6(x_{k+1}-x_k)}\textcolor{red}{M_{k+1}} \\
& + \left(y_k - \frac{\textcolor{red}{M_k}(x_k-x_{k+1})^2}{6}\right)\frac{x-x_{k+1}}{x_k-x_{k+1}} \\
& + \left(y_{k+1} - \frac{\textcolor{red}{M_{k+1}}(x_{k+1}-x_k)^2}{6}\right)\frac{x-x_k}{x_{k+1}-x_k}
\end{align*}

对上式求导,得到一阶导数表达式
\begin{align*}
s_k'(x) = & \frac{M_k}{6}\frac{3(x-x_{k+1})^2 - (x_k-x_{k+1})^2}{x_k-x_{k+1}} \\
& + \frac{M_{k+1}}{6}\frac{3(x-x_k)^2-(x_{k+1}-x_k)^2}{x_{k+1}-x_k} \\
& + f[x_k, x_{k+1}]
\end{align*}

\begin{align*}
\Rightarrow & s_k'(x_k) = \frac{2M_k + M_{k+1}}{6}(x_k-x_{k+1}) + f[x_k, x_{k+1}] \\
& s_k'(x_{k+1}) = \frac{2M_{k+1} + M_k}{6}(x_{k+1}-x_k) + f[x_k, x_{k+1}]
\end{align*}

代入条件 $s_k'(x_{k+1})=s_{k+1}'(x_{k+1})$\\
并利用 $h_k = x_{k+1} - x_k$ 化简,得
\begin{align*}
h_k\textcolor{red}{M_k} + 2(h_k + h_{k+1})\textcolor{red}{M_{k+1}} + h_{k+1}\textcolor{red}{M_{k+2}} \\
= 6(f[x_{k+1}, x_{k+2}]-f[x_k, x_{k+1}])
\end{align*}

$k = 0, 1, \ldots, n - 2$

\textbf{误差估计}
设 $f(x) \in C^4[a,b]$, $S(x)$ 为满足第一或第二类边界条件的三次样条函数,则有
\begin{align*}
\max_{a\leq x\leq b}|f(x)-S(x)| &\leq \frac{5}{384}\max_{a\leq x\leq b}|f^{(4)}(x)|h^4 \\
\max_{a\leq x\leq b}|f'(x)-S'(x)| &\leq \frac{1}{24}\max_{a\leq x\leq b}|f^{(4)}(x)|h^3 \\
\max_{a\leq x\leq b}|f''(x)-S''(x)| &\leq \frac{3}{8}\max_{a\leq x\leq b}|f^{(4)}(x)|h^2
\end{align*}
其中 $h = \max\limits_{0\leq k\leq n-1}h_k = \max\limits_{0\leq k\leq n-1}|x_{k+1}-x_k|$.

\section*{3-函数逼近}

\subsection*{①\ 最佳一致逼近}

\textbf{偏差点}
\[
|p_n(x_0) - f(x_0)| = \|p_n - f\|_{\infty}
\]

\textbf{Chebyshev定理}
$p_n(x) \in H_n$ 是 $f(x) \in C[a, b]$ 的最佳一致逼近多项式的 \textbf{充分必要条件} 是 $p_n(x)$ 在 $[a, b]$ 上有 $n+2$ 个轮流为"正"、"负"的偏差点

\begin{itemize}
\item \textbf{推论1}:若 $f(x) \in C[a, b]$,则其 $n$ 次最佳一致逼近多项式存在且唯一
\item \textbf{推论2}:若 $f(x) \in C[a, b]$,则其 $n$ 次最佳一致逼近多项式是 $f(x)$ 的一个 Lagrange 插值多项式
\end{itemize}

\subsection*{②\ 最佳平方逼近}

\textbf{正交多项式}
\begin{itemize}
\item \textbf{性质 1}:$\{\phi_k(x)\}_{k=0}^{n}$ 构成 $H_n$ 的一组基
\item \textbf{性质 2}:$\phi_n(x)$ 与所有次数小于 $n$ 的多项式正交
\item \textbf{性质 3}:三项递推公式 设 $\phi_k$ 首项系数为 $a_k \neq 0$,则有
\begin{align*}
\textcolor{blue}{\phi_{n+1}(x)} = & \frac{a_{n+1}}{a_n}\left[\left(x - \frac{\langle x\phi_n(x), \phi_n(x) \rangle}{\langle \phi_n(x), \phi_n(x) \rangle}\right)\textcolor{blue}{\phi_n(x)}\right. \\
& \left. - \frac{a_{n-1}}{a_n}\frac{\langle \phi_n(x), \phi_n(x) \rangle}{\langle \phi_{n-1}(x), \phi_{n-1}(x) \rangle}\textcolor{blue}{\phi_{n-1}(x)}\right]
\end{align*}
\item \textbf{性质 4}:设 $\{\phi_k(x)\}_{k=0}^{\infty}$ 为 $[a, b]$ 上的带权 $\rho(x)$ 的 $k$ 次正交多项式,则 $\phi_n(x)$ 在 $(a, b)$ 内有 $n$ 个 \textcolor{red}{不同的零点(单实根)}
\end{itemize}

\fbox{\textbf{法方程(也适用于非正交多项式的基)}}
\begin{align*}
& \left\langle f - \sum_{j=0}^{n} a_j \phi_j, \phi_k \right\rangle = 0 \\
\Rightarrow & \colorbox{blue!10}{$\displaystyle\sum_{j=0}^{n}\langle \phi_k, \phi_j \rangle a_j = \langle f, \phi_k \rangle$}
\end{align*}

\textbf{求解 $n$ 次最佳平方逼近多项式}

\paragraph{方法 1}:以 $1, x, x^2, \ldots, x^n$ 为基,设 $\rho(x) \equiv 1$, $f(x) \in C[0, 1]$
直接求解法方程  
$\sum_{j=0}^{n}\langle \phi_k, \phi_j \rangle \textcolor{red}{a_j} = \langle f, \phi_k \rangle$, $k = 0, 1, \ldots, n$\\
其中 $\langle \phi_k, \phi_j \rangle = \int_0^1 x^{k+j} dx = \frac{1}{k+j+1}$\\
最后代入 $p^*(x) = \sum_{j=0}^{n} a_j \phi_j(x)$ 即可

\colorbox{green!20}{$
\langle f - p^*, \phi_k \rangle = 0,\quad k = 0, 1, \ldots, n
$}

\textbf{平方误差}:
\begin{align*}
\|\delta(x)\|_2^2 &= \langle f - p^*, f - p^* \rangle \\
&= \langle f, f \rangle - \langle p^*, f \rangle \\
&= \|f(x)\|_2^2 - \sum_{j=0}^{n}\textcolor{red}{a_j}\langle \phi_j, f \rangle
\end{align*}

\paragraph{方法 2}:以正交多项式 $\phi_0, \phi_1, \phi_2, \ldots, \phi_n$ 为基\\
\textbf{平方误差}:$\|\delta(x)\|_2^2 = \|f(x)\|_2^2 - \sum_{j=0}^{n}\textcolor{red}{\dfrac{\langle f, \phi_j \rangle^2}{\langle \phi_j, \phi_j \rangle}}$

\columnbreak

\textbf{常见的正交多项式}

\subsubsection*{1. Legendre 多项式}
当区间为 $[-1, 1]$,权函数 $\rho(x) \equiv 1$ 时,由 $\{1, x, x^2, \ldots, x^n, \ldots\}$ 正交化得到的多项式
\[
P_n(x) = \frac{1}{2^n n!} \frac{d^n}{dx^n}[(x^2-1)^n]
\]

\begin{align*}
P_0(x) &= 1 \\
P_1(x) &= x \\
P_2(x) &= \frac{3x^2-1}{2} \\
P_3(x) &= \frac{5x^3 - 3x}{2} \\
P_4(x) &= \frac{35x^4-30x^2+1}{8}
\end{align*}

\textbf{性质}
\begin{enumerate}
\item 递推关系:$(n+1)P_{n+1}(x) = (2n+1)xP_n(x) - nP_{n-1}(x)$
\item 正交性:$\|P_n(x)\|_2^2 = \dfrac{2}{2n+1}$
\item 奇偶性
\end{enumerate}

\begin{itemize}
\item \textbf{一般区间上的最佳平方逼近多项式}\\
思路:变量代换\\
$g(x) = f(\mu + \delta x)$, $p_f(x) = p_g\left(\dfrac{x-\mu}{\delta}\right)$
\end{itemize}

\subsubsection*{2. Chebyshev 多项式}
当区间为 $[-1, 1]$,权函数 $\rho(x) = \dfrac{1}{\sqrt{1-x^2}}$ 时,由 $\{1, x, x^2, \ldots, x^n, \ldots\}$ 正交化得到的多项式
\[
T_n(x) = \cos(n \cdot \arccos(x)),\quad |x| \leq 1
\]
若令 $x = \cos\theta$,则 $T_n(x) = \cos n\theta$

\begin{align*}
T_0(x) &= 1 \\
T_1(x) &= x \\
T_2(x) &= 2x^2 - 1 \\
T_3(x) &= 4x^3 - 3x \\
T_4(x) &= 8x^4 - 8x^2 + 1
\end{align*}

\textbf{性质}
\begin{enumerate}
\item 递推关系:$T_{n+1}(x) = 2xT_n(x) - T_{n-1}(x)$\\
$T_n(x)$ 的最高项系数为 $2^{n-1}$ ($n \geq 1$)
\item 正交性:$\|T_n(x)\|_2^2 = \begin{cases}
\frac{\pi}{2}, & n \neq 0 \\
\pi, & n = 0
\end{cases}$
\item 奇偶性
\item $n$个零点:$\cos\left(\dfrac{2k-1}{2n}\pi\right)$, $k = 1, 2, \ldots, n$\\
\colorbox{red!10}{\textbf{Chebyshev节点}}
\item $n+1$个极值点:$\cos\dfrac{k\pi}{n}$, $k = 0, 1, 2, \ldots, n$
\end{enumerate}

\textbf{定理}
在区间 $[-1,1]$ 上的所有最高项系数为 $1$ 的 $n$ 次多项式中,$\dfrac{1}{2^{n-1}}T_n(x)$ \textcolor{red}{与零的偏差最小},其偏差为 $\dfrac{1}{2^{n-1}}$

由 Chebyshev 定理可知,在区间 $[-1, 1]$ 上,$x^n$ 在 $H_{n-1}$ 中的最佳一致逼近多项式是 $x^n - \dfrac{1}{2^{n-1}}T_n(x)$

\bigbreak

\textbf{定理}
设函数 $f(x) \in C^{n+1}[-1, 1]$,若插值节点 $x_0, x_1, \ldots, x_n$ 为 Chebyshev 多项式的 $n+1$ 个零点,则插值误差 $\|f(x) - L_n(x)\|_{\infty}$ 最小,且
\[
\|f(x) - L_n(x)\|_{\infty} = \max_{-1\leq x\leq 1}|R_n(x)| \leq \frac{|M_{n+1}|}{(n+1)!}\|\omega_{n+1}(x)\|_{\infty} = \frac{|M_{n+1}|}{2^n(n+1)!}
\]

\colorbox{green!20}{%
\begin{minipage}{\linewidth}
• 用 Chebyshev 多项式的零点插值,可以使得总体插值误差最小!\\
• 可通过 \textcolor{blue}{变量代换},将上述定理推广至区间 $[a, b]$
\end{minipage}
}

\subsubsection*{3. Hermite 多项式}
当区间为 $(-\infty, \infty)$,权函数 $\rho(x) = e^{-x^2}$ 时,由 $\{1, x, x^2, \ldots, x^n, \ldots\}$ 正交化得到的多项式
\[
H_n(x) = (-1)^n e^{x^2} \frac{d^n}{dx^n}e^{-x^2} = e^{-x^2} \frac{d^n}{dx^n}e^{x^2}
\]

\begin{align*}
H_0(x) &= 1 \\
H_1(x) &= 2x \\
H_2(x) &= 4x^2 - 2 \\
H_3(x) &= 8x^3 - 12x \\
H_4(x) &= 16x^4 - 48x^2 + 12
\end{align*}

\textbf{性质}
\begin{itemize}
\item 递推关系:$H_{n+1}(x) = 2xH_n(x) - 2nH_{n-1}(x)$ \\
首项系数为 $2^n$
\item 正交性:$\|H_n(x)\|_2^2 = \sqrt{\pi}2^n n!$
\end{itemize}

\colorbox{green!20}{%
\begin{minipage}{\linewidth}
Hermite 多项式可用作神经网络的激活函数的一组基
\end{minipage}
}

\subsection*{③ 曲线拟合的最小二乘法}

\textbf{求解最小二乘逼近函数}
\begin{itemize}
\item \textbf{方法 1}:直接求解法方程
\item \textbf{方法 2}:用正交多项式作最小二乘拟合
\end{itemize}

\textbf{构造带权正交多项式的方法}
\begin{itemize}
\item \textbf{三项递推公式}
\[
\left\{
\begin{aligned}
\phi_0(x) &= 1,\quad \phi_1(x) = x - \alpha_0, \\
\phi_{k+1}(x) &= (x - \alpha_k)\phi_k(x) - \beta_k \phi_{k-1}(x),
\end{aligned}
\right.
\]
其中
\begin{align*}
\alpha_k &= \frac{\langle x\phi_k, \phi_k \rangle}{\langle \phi_k, \phi_k \rangle} \\
\beta_k &= \frac{\langle \phi_k, \phi_k \rangle}{\langle \phi_{k-1}, \phi_{k-1} \rangle}
\end{align*}

\item \textbf{正交多项式是目前为止多项式拟合的最好方法}\\
不需要解线性方程组,给定次数 $n$,可以根据三项递推公式方便地计算正交多项式
\end{itemize}

\end{multicols}
\end{document}